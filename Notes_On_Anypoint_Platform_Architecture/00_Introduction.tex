%%%%%%%%%%%%%%%%%%%%%%%%%%%%%%%%%%%%%%%%%
% Formal Book Title Page
% LaTeX Template
% Version 2.0 (23/7/17)
%
% This template was downloaded from:
% http://www.LaTeXTemplates.com
%
% Original author:
% Peter Wilson (herries.press@earthlink.net) with modifications by:
% Vel (vel@latextemplates.com)
%
% License:
% CC BY-NC-SA 3.0 (http://creativecommons.org/licenses/by-nc-sa/3.0/)
%
% This template can be used in one of two ways:
%
% 1) Content can be added at the end of this file just before the \end{document}
% to use this title page as the starting point for your document.
%
% 2) Alternatively, if you already have a document which you wish to add this
% title page to, copy everything between the \begin{document} and
% \end{document} and paste it where you would like the title page in your
% document. You will then need to insert the packages and document
% configurations into your document carefully making sure you are not loading
% the same package twice and that there are no clashes.
%
%%%%%%%%%%%%%%%%%%%%%%%%%%%%%%%%%%%%%%%%%

%----------------------------------------------------------------------------------------
% PACKAGES AND OTHER DOCUMENT CONFIGURATIONS
%----------------------------------------------------------------------------------------

\documentclass[a4paper, 11pt, oneside]{book} % A4 paper size, default 11pt font size and oneside for equal margins

\newcommand{\plogo}{\fbox{$\mathcal{PL}$}} % Generic dummy publisher logo

\usepackage[utf8]{inputenc} % Required for inputting international characters
\usepackage[T1]{fontenc} % Output font encoding for international characters
\usepackage{fouriernc} % Use the New Century Schoolbook font
\usepackage{hyperref}
\usepackage{geometry}
\geometry{
	a4paper,
	total={170mm,257mm},
	left=25mm,
	top=20mm,
}
\usepackage{graphicx}
\graphicspath{ {.} }
\usepackage{caption}

\usepackage{calc}

\usepackage{mdframed}
\usepackage{fancybox}

%----------------------------------------------------------------------------------------
% TITLE PAGE
%----------------------------------------------------------------------------------------



\begin{document}

\begin{titlepage} % Suppresses headers and footers on the title page

 \centering % Centre everything on the title page

 \scshape % Use small caps for all text on the title page

 \vspace*{\baselineskip} % White space at the top of the page

 %------------------------------------------------
 % Title
 %------------------------------------------------

 \rule{\textwidth}{1.6pt}\vspace*{-\baselineskip}\vspace*{2pt} % Thick horizontal rule
 \rule{\textwidth}{0.4pt} % Thin horizontal rule

 \vspace{0.75\baselineskip} % Whitespace above the title

 {\LARGE Welcome To Anypoint Platform Architecture: \\
 		Application Networks \\ \small\LaTeX ~} % Title

 \vspace{0.75\baselineskip} % Whitespace below the title

 \rule{\textwidth}{0.4pt}\vspace*{-\baselineskip}\vspace{3.2pt} % Thin horizontal rule
 \rule{\textwidth}{1.6pt} % Thick horizontal rule

 \vspace{2\baselineskip} % Whitespace after the title block

 %------------------------------------------------
 % Subtitle
 %------------------------------------------------

 % Subtitle or further description

 \vspace*{3\baselineskip} % Whitespace under the subtitle

 %------------------------------------------------
 % Editor(s)
 %------------------------------------------------

 Edited By

 \vspace{0.5\baselineskip} % Whitespace before the editors

 {\scshape\Large Boyuan Zhang \\} % Editor list

 \vspace{0.5\baselineskip} % Whitespace below the editor list

 \textit{} % Editor affiliation

 \vfill % Whitespace between editor names and publisher logo

 %------------------------------------------------
 % Publisher
 %------------------------------------------------

 %\plogo % Publisher logo

 \vspace{0.3\baselineskip} % Whitespace under the publisher logo

 2019 % Publication year

  {\large MuleSoft Inc.} % Publisher

\end{titlepage}

%----------------------------------------------------------------------------------------

\newpage




\tableofcontents

%\part{Week 1}

\chapter{Introduction}

\section{There are two architecture courses and certifications}

\begin{itemize}

	\item Anypoint Platform Architecture: \textbf{Application Networks} and MuleSoft Certified \textbf{Platform Architect} - Level 1

	\begin{itemize}
		\item Define and be responsible for an organization's Anypoint Platform strategy
		\item Direct the emergence of an effective application network out of individual integration solutions following API-led connectivity across an organization
	\end{itemize}

	\item Anypoint Platform Architecture: \textbf{Integration Solutions} and MuleSoft Certified \textbf{Integration Architect} - Level 1

	\begin{itemize}
		\item Drive and be responsible for an organization's Anypoint Platform implementation and the technical quality, governance (ensuring compliance), and operationalization of the integration solutions.
		\item Work with technical and non-technical stakeholders to translate functional and non-functional requirements into integration interfaces and implementations
	\end{itemize}

\end{itemize}

\textbf{Prerequisites}\\

\textbf{Experience with Anypoint Platform} and its constituent components

\begin{itemize}
	\item \textit{Getting Started} with \textit{Anypoint Platform}
	\item \textit{Anypoint Platform Development}: Fundamentals
	\item \textit{MuleSoft.U development Fundamentals}
	\item \textit{API-Led connectivity Workshop} by MuleSoft Presales upon request
\end{itemize}

\begin{mdframed}[backgroundcolor=yellow]
	The \texttit{target audience} of this course are architects, especially Enterprise Architects and Solution Architects, new to Anypoint Platform, API-led connectivity and the application network approach, but experienced in other
\end{mdframed}







\end{document}
